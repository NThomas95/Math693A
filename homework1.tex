\documentclass[11pt]{article}

\usepackage{amsmath}
\usepackage{amssymb}
\usepackage[margin=1 in]{geometry}
\usepackage{graphicx}
\usepackage{float}
\usepackage{color}
\usepackage{subcaption}
\usepackage{wrapfig}
\newtheorem{lemma}{Lemma}
\usepackage{datatool}
\usepackage{siunitx}
\usepackage{booktabs} % for nicer tables
\usepackage{caption} % improve caption spacing (among other things)
\usepackage{bm}
\renewcommand*\dtldisplaystarttab{\toprule}
\renewcommand*\dtldisplayendtab{\tabularnewline\bottomrule}
\renewcommand*\dtldisplayafterhead{\midrule}

\begin{document}
\title{Homework 1}
\author{Nathan Thomas}
\markboth{Math 693A Homework 1}{}
\date{September 21, 2018}
\maketitle
\section{Problem}
$\text{NW}^{2}-3.1$: Program the steepest decent and newton algorithms to use the backtracking line search. Use them to minimize the Rosenbrock function

\[
f(\bar{x}) = 100(x_{2}-x_{1}^{2})^{2} + (1-x_{1})^{2}
\]

Set the initial step length $\alpha_{0}=1$ and report the step length used by each method at each iteration. First try the initial point $\bar{x}_{0}=[1.2,1.2]^{T}$ and then the more difficult starting point $\bar{x}_{0}=[-1.2,1]^{T}$.\\

Suggested values: $alpha_{0}=1, \ \rho=\frac{1}{2}, \ c=10^{-4}$. Stop when $|f(\bar{x}_{k}|<10^{-8}$ or $||\nabla f(\bar{x}_{k}||<10^{-8}$S

\section{Results}
\subsection{Steepest Decent}
The steepest decent direction converges regardless of the starting point.
\subsubsection{Case 1, $\bar{x}_{0}=[1.2,1.2]^{T}$}
\DTLloaddb
[
noheader,
keys={x,y,theta},
headers={
	\shortstack{$\boldsymbol{\theta_{{2,i}}}$},
	\shortstack{X},
	\shortstack{Y}}
]
{steepest_decent_1}{steepest_decent_1.csv}
\begin{table}[t]
	\sisetup{
		parse-numbers   = false,
		table-number-alignment = left,
		table-figures-integer = 4,
		table-figures-decimal = 4,
		input-decimal-markers = .
	}
	\renewcommand*\dtlrealalign{S}
	\caption{Database file}
	\centering
	\DTLdisplaydb{steepest_decent_1}
\end{table}
\subsubsection{Case 1, $\bar{x}_{0}=[-1.2,1]^{T}$}

\subsection{Newton Direction} 
The Newton Direction only converged with the first starting point.
\subsubsection{Case 1, $\bar{x}_{0}=[1.2,1.2]^{T}$}

\subsubsection{Case 1, $\bar{x}_{0}=[-1.2,1]^{T}$}




\end{document}